\documentclass[11pt,letterpaper,sans]{moderncv}        % possible options include font size ('10pt', '11pt' and '12pt'), paper size ('a4paper', 'letterpaper', 'a5paper', 'legalpaper', 'executivepaper' and 'landscape') and font family ('sans' and 'roman')

% moderncv themes
\moderncvstyle{classic}                             % style options are 'casual' (default), 'classic', 'oldstyle' and 'banking'
\moderncvcolor{blue}                               % color options 'blue' (default), 'orange', 'green', 'red', 'purple', 'grey' and 'black'
%\renewcommand{\familydefault}{\sfdefault}         % to set the default font; use '\sfdefault' for the default sans serif font, '\rmdefault' for the default roman one, or any tex font name
\nopagenumbers{}


% adjust the page margins
\usepackage[scale=0.75]{geometry}
%\setlength{\hintscolumnwidth}{3cm}                % if you want to change the width of the column with the dates
%\setlength{\makecvtitlenamewidth}{10cm}           % for the 'classic' style, if you want to force the width allocated to your name and avoid line breaks. be careful though, the length is normally calculated to avoid any overlap with your personal info; use this at your own typographical risks...

% personal data
\name{Clinton}{Dreisbach}
\address{Durham, NC}
\phone[mobile]{+1~(919)~360~9150}
\email{clinton@dreisbach.us}
\homepage{dreisbach.us}
\social[linkedin]{cndreisbach}
\social[twitter]{cndreisbach}
\social[github]{cndreisbach}


%----------------------------------------------------------------------------------
%            content
%----------------------------------------------------------------------------------

\begin{document}
	\makecvtitle

	\section{Summary}
	I am a senior developer and development manager with more than 17 years of experience, focusing on web development and data science.
	\par\bigskip
	My passion is teaching and mentoring other developers. When not working as a teacher, I volunteer for community events like ClojureBridge, local Teen Tech Camps, and in public schools.


	\section{Skills}
	\cvitem{Expert}{Python, Ruby, SQL}
	\cvitem{Proficient}{Clojure, Perl, JavaScript, HTML, Django, Flask, React, API design}
	\cvitem{Competent}{Machine learning, Pandas, NumPy, Scheme, CSS}
	\cvitem{Beginner}{\LaTeX, interpreters}
	\cvitem{Want to learn}{Common Lisp, compiler design}


	\section{Experience}
	\cventry{Aug 2015 -- present}{Software Development Manager}{RTI International}{Durham, NC}{}{
		Led development efforts in RTI's Center for Data Science, using Python, JavaScript, D3, and React to build dynamic web applications for data analysis.
		\bigskip
		\begin{itemize}%
			\item Developed a dashboard used to track disease at the 2015 Kumbh Mela, one of the world's largest outdoor gatherings.
			\item Created two open-source applications, Call for Service Analytics and Statistical Traffic Analysis Report, for assisting law enforcement with understanding 911 call data and finding evidence of racial bias in traffic stop data.
		\end{itemize}
	}
	\bigskip	
	
	\cventry{Apr 2014 -- Aug 2015}{Programming Instructor}{The Iron Yard}{Durham, NC}{}{
		I taught intensive 12-week courses in Python and data science to beginning programmers, equipping them with the knowledge and tools they needed to succeed in their careers.
		\bigskip
		\begin{itemize}%
			\item Created a Python and data science curriculum.
			\item Developed a Ruby and Rails curriculum.
			\item Placed students with companies such as Red Hat, Intuit, and Automated Insights.
		\end{itemize}
	}
	\bigskip

	\cventry{Nov 2012 -- Apr 2014}{Technology \& Innovation Fellow}{Consumer Financial Protection Bureau}{\mbox{Washington, DC}}{}{
		Built systems to help American consumers and to improve government.
		\bigskip
		\begin{itemize}
			\item Designed and created \href{https://github.com/cfpb/qu}{Qu}, a data platform used to serve public datasets. Qu is capable of streaming datasets of arbitrary length, ingesting and manipulating datasets based off definition files, and batch processing map-reduce jobs on that data. Qu is powered by Clojure and MongoDB.
			\item Co-chair of the Open Source Working Group. Helped coordinate a major effort to open source over a dozen projects.
			\item Helped organize an internal testing mini-conference aimed at increasing testing on CFPB projects. Within a year, all projects were using continuous integration and had comprehensive test suites.
		\end{itemize}
	}
	
	\bigskip

	\cventry{Aug 2011 -- Nov 2012}{Software Developer}{Relevance (now Cognitect)}{Durham, NC}{}{
		Designed, created, and improved large software systems in Ruby and Clojure. Created \href{https://github.com/relevance/diametric}{Diametric}, a Ruby ActiveModel-compliant interface to the Datomic database.
	}

	\cventry{Aug 2007 -- Mar 2011}{Development Director}{Viget Labs}{Durham, NC}{}{
		Led Durham, NC and Washington, DC based development teams. Worked as lead developer on Ruby on Rails projects for established businesses and startup clients.
	}

	\cventry{Sep 2005 -- Jul 2007}{Senior Developer}{Lulu.com}{Morrisville, NC}{}{
		Led development team for community features at this startup, meeting the needs of hundreds of thousands of users.
	}

	\cventry{July 2003 -- Aug 2005}{Assistant Vice-President}{ShowMaster Systems}{New Orleans, LA}{}{
		Managed department of developers and information analysts and oversaw the administration of multi-million dollar trade shows.
	}

	\cventry{Oct 1999 -- July 2003}{Contractor}{}{Washington, DC and Seattle, WA}{}{
		Worked as a contractor, using Perl and PHP, for organizations such as Boeing, Real Networks, AT\&T Wireless, and various government agencies.
	}

	
	\cventry{Oct 1995 -- Oct 1999}{Signals intelligence analyst}{US Army}{}{}{
		Analyzed electronic communications for patterns to convey intelligence about foreign military actions. Created software in Tcl and Perl to assist in analysis.
	}
	
	\section{Volunteer Projects}
	\cventry{2014}{ClojureBridge}{}{}{}{
\href{http://clojurebridge.org/}{ClojureBridge} is an initiative to increase diversity within the Clojure community by offering free, beginner-friendly workshops for women. I was the lead author of the core curriculum, and taught the first workshop.}

	\cventry{2013}{Citizen Schools}{}{}{}{
I taught Scheme using the Bootstrap curriculum through the Citizen Schools after-school program for 6th-8th graders.}

	\cventry{2013}{Teen Tech Camp}{}{}{}{
Teen Tech Camp was a day-long event that members of Refresh the Triangle and the Triangle Python Users' Group organized with the help of the Durham County Library. We successfully applied for a grant from the Python Software Foundation to purchase Raspberry Pis for 20 students, aged 12-18. I was an organizer and lead teacher.
	}

\end{document}
